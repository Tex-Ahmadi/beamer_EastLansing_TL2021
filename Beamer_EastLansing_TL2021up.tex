% M.ahmadi 1395.11.28   edit: 1400.11.15
% @Tex_Ahmadi 
% تست شده با تکلایو 2021
% دو بار پردازش با زی‌لاتک
\PassOptionsToPackage{pdfpagemode=FullScreen,hyperfootnotes=false}{hyperref}
\documentclass[10pt,xcolor=dvipsnames]{beamer}

%\usepackage{amsmath,amssymb,amsfonts} 
\usepackage{tikz} 
\usepackage{graphicx}
\usepackage{listings}
\usepackage{ptext}

\usetheme{EastLansing}
\usefonttheme{serif,professionalfonts}
%\usecolortheme[named=blue]{structure}
\setbeamercovered{transparent}
 %توجه: بسته هایی که نیاز هست قبل از بسته زی‌پرشین نوشته شود.
\usepackage{xepersian}
\settextfont{XB Zar}
\setlatintextfont{Times New Roman}
%در نسخه اولیه 2021 نیاز هست.
\ExplSyntaxOn
\cs_set_eq:NN
\etex_iffontchar:D
\tex_iffontchar:D
\cs_undefine:N \c_one
\int_const:Nn \c_one { 1 }
\ExplSyntaxOff
\setdigitfont{Yas}

\defpersianfont\nas[Scale=1.5]{IranNastaliq}
\defpersianfont\xb[Scale=1.3]{XB Zar}
\deflatinfont\tnr[Scale=1.2]{Times New Roman}
%\linespread{1.2} 
% دستورات مورد نیاز برای استفاده از کلاس بیمر در command نوشته شده
% @Tex_Ahmadi    Edit:1400.07.27  test in TL2021
% M.ahmadi 1395.11.08, copy of:
% http://qa.parsilatex.com/14100
% http://qa.parsilatex.com/14148
%%%%%%%%%%%%%%%%%
\makeatletter
\expandafter\let\csname beamer@@tmpop@itemize item@default\endcsname\relax
\expandafter\let\csname beamer@@tmpop@itemize subitem@default\endcsname\relax
\expandafter\let\csname beamer@@tmpop@itemize subsubitem@default\endcsname\relax

\defbeamertemplate*{itemize item}{default}{\scriptsize\raise1.25pt\hbox{\donotcoloroutermaths$\if@RTL\blacktriangleleft \else \blacktriangleright\fi$}}
\defbeamertemplate*{itemize subitem}{default}{\tiny\raise1.5pt\hbox{\donotcoloroutermaths$\if@RTL\blacktriangleleft \else \blacktriangleright\fi$}}
\defbeamertemplate*{itemize subsubitem}{default}{\tiny\raise1.5pt\hbox{\donotcoloroutermaths$\if@RTL\blacktriangleleft \else \blacktriangleright\fi$}}

\bidi@patchcmd{\@listi}{\leftmargin}{\rightmargin}{}{}
\let\@listI\@listi
\bidi@patchcmd{\@listii}{\leftmargin}{\rightmargin}{}{}
\bidi@patchcmd{\@listiii}{\leftmargin}{\rightmargin}{}{}
\bidi@patchcmd{\beamer@enum@}{\raggedright}{\raggedleft}{}{}
\bidi@patchcmd{\@@description}{\raggedright}{\raggedleft}{}{}
\bidi@patchcmd{\@@description}{\leftmargin}{\rightmargin}{}{}

\renewcommand{\itemize}[1][]{%
  \beamer@ifempty{#1}{}{\def\beamer@defaultospec{#1}}%
  \ifnum \@itemdepth >2\relax\@toodeep\else
    \advance\@itemdepth\@ne
    \beamer@computepref\@itemdepth% sets \beameritemnestingprefix
    \usebeamerfont{itemize/enumerate \beameritemnestingprefix body}%
    \usebeamercolor[fg]{itemize/enumerate \beameritemnestingprefix body}%
    \usebeamertemplate{itemize/enumerate \beameritemnestingprefix body begin}%
    \list
      {\usebeamertemplate{itemize \beameritemnestingprefix item}}
      {\def\makelabel##1{%
          {%
            \hss\llap{{%
                \usebeamerfont*{itemize \beameritemnestingprefix item}%
                \usebeamercolor[fg]{itemize \beameritemnestingprefix item}##1}}%
          }%
        }%
      }
  \fi%
  \beamer@cramped%
  \raggedleft%
  \beamer@firstlineitemizeunskip%
}
\makeatother
\hypersetup{pdfinfo={Author:={M.Ahmadi 1400-07-27}}}
%%%%%%%%%%%%%%%%%%
\makeatletter
\bidi@undef\beamer@@tmpop@footnote@default
\defbeamertemplate*{footnote}{default}
{
  \parindent 1em\noindent%
  \raggedleft
  \hbox to 1.8em{\hfil\insertfootnotemark}\insertfootnotetext\par%
}
\defbeamertemplate*{LTRfootnote}{default}
{
  \parindent 1em\noindent%
  \raggedright
  \hbox to 1.8em{\hfil\insertfootnotemark}\latinfont\insertfootnotetext\par%
}
\footdir@temp\footdir@ORG@bidi@beamer@framefootnotetext\beamer@framefootnotetext{R}
\let\@footnotetext=\beamer@framefootnotetext
\let\@RTLfootnotetext\@footnotetext
\def\@makeLTRfntext#1{%
  \def\insertfootnotetext{#1}%
  \def\insertfootnotemark{\@makefnmark}%
  \usebeamertemplate***{LTRfootnote}}
\newcommand<>\beamer@frameLTRfootnotetext[1]{%
  \global\setbox\beamer@footins\vbox{\@RTLfalse%
    \hsize\framewidth
    \textwidth\hsize
    \columnwidth\hsize
    \unvbox\beamer@footins
    \reset@font\footnotesize
    \@parboxrestore
    \protected@edef\@currentlabel
         {\csname p@footnote\endcsname\@thefnmark}%
    \color@begingroup
      \uncover#2{\@makeLTRfntext{%
        \rule\z@\footnotesep\ignorespaces#1\@finalstrut\strutbox}}%
    \color@endgroup}}
\footdir@temp\footdir@ORG@bidi@beamer@frameLTRfootnotetext\beamer@frameLTRfootnotetext{L}
\let\@LTRfootnotetext=\beamer@frameLTRfootnotetext
\makeatother
%%%%%%%%%%%%%%%
\makeatletter
\long\def\beamer@newenvnoopt#1#2#3#4{%
  \expandafter\renewcommand\expandafter<\expandafter>\csname#1\endcsname[#2]{#3}%<- here
  \expandafter\long\expandafter\def\csname end#1\endcsname{#4}%
}
\long\def\beamer@newenvopt#1#2[#3]#4#5{%
  \expandafter\renewcommand\expandafter<\expandafter>\csname#1\endcsname[#2][#3]{#4}%<- here
  \expandafter\long\expandafter\def\csname end#1\endcsname{#5}%
}
\renewcommand<>\beamer@columncom[2][\beamer@colmode]{%
  \beamer@colclose%
  \def\beamer@colclose{\end{minipage}\hfill\end{actionenv}\ignorespaces}%
\begin{actionenv}#3%
  \setkeys{beamer@col}{#1}%
  \begin{minipage}[\beamer@colalign]{#2}%
    \leavevmode\raggedleft\beamer@colheadskip\ignorespaces}
\renewenvironment<>{columns}[1][]{%
  \begin{actionenv}#2%
  \def\beamer@colentrycode{%
    \hbox to\textwidth\bgroup%
    \leavevmode%
    \hskip-\beamer@leftmargin%
    \nobreak%
    \beamer@tempdim=\textwidth%
    \advance\beamer@tempdim by\beamer@leftmargin%
    \advance\beamer@tempdim by\beamer@rightmargin%
    \hbox to\beamer@tempdim\bgroup%
    \hbox{}\hfill\ignorespaces}%
  \def\beamer@colexitcode{\egroup%
    \nobreak%
    \hskip-\beamer@rightmargin\egroup}%
  \ifbeamer@centered\setkeys{beamer@col}{c}\else\setkeys{beamer@col}{t}\fi%
  \setkeys{beamer@col}{#1}%
  \par%
  \leavevmode\beamer@colentrycode%
  \def\beamer@colclose{}\ignorespaces}%
  {\beamer@colclose\def\beamer@colclose{}\beamer@colexitcode\end{actionenv}}%
\makeatother
%%%%%%%%%%%%%%%%%
\makeatletter
\expandafter\let\csname beamer@@tmpop@section in toc@ball\endcsname\relax
\defbeamertemplate{section in toc}{ball}
{\leavevmode\rightskip=2.75ex%
  \llap{%
    \normalsize%
    \begin{pgfpicture}{-1ex}{-0.7ex}{1ex}{1ex}
      \pgftext{\beamer@usesphere{section number projected}{tocsphere}}
      \pgftext{%
        \usebeamerfont*{section number projected}%
        \usebeamercolor{section number projected}%
        \color{fg!90!bg}%
        \inserttocsectionnumber}
    \end{pgfpicture}%
    \kern1.25ex}%
  \inserttocsection\par
}
[action]
{\setbeamerfont{section number projected}{size=\scriptsize}}
\expandafter\let\csname beamer@@tmpop@subsection in toc@ball\endcsname\relax
\defbeamertemplate{subsection in toc}{ball}
{\leavevmode\rightskip=5ex%
  \llap{\raise0.1ex\beamer@usesphere{subsection number projected}{bigsphere}\kern1ex}%
  \inserttocsubsection\par%
}
\expandafter\let\csname beamer@@tmpop@subsubsection in toc@ball\endcsname\relax
\defbeamertemplate{subsubsection in toc}{ball}
{\leavevmode\normalsize\usebeamerfont{subsection in
    toc}\rightskip=7ex\usebeamerfont{subsubsection in toc}%
  \llap{\beamer@usesphere{subsubsection number projected}{bigsphere}\kern0.75ex}%
  \inserttocsubsubsection\par%
}
\setbeamertemplate{sections/subsections in toc}[ball]
\makeatother
\defpersianfont\nast[Scale=1.5,ExternalLocation]{IranNastaliq} 
%+++++++++++++++++++++++++++
\AtBeginEnvironment{itemize}{\vskip0pt}
\AtBeginEnvironment{enumerate}{\vskip0pt}
\AtBeginEnvironment{description}{\vskip0pt}
% متلب
\definecolor{mygreen}{RGB}{28,172,0} 
\definecolor{mylilas}{RGB}{170,55,241}
\lstset{language=Matlab,
    breaklines=true,basicstyle=\ttfamily\scriptsize,
    morekeywords={matlab2tikz},
    keywordstyle=\color{blue},
    morekeywords=[2]{1}, keywordstyle=[2]{\color{black}},
    identifierstyle=\color{black},
    stringstyle=\color{mylilas},
    commentstyle=\color{mygreen},
    showstringspaces=false
}
%++++++++++++
%برای شماره خوردن قضیه،...
\setbeamertemplate{theorems}[numbered]
%برای شماره خوردن عنوان جدول و شکل
\setbeamertemplate{caption}[numbered]
%محیط های قضیه، تعریف، مثال، لم، نتیجه و.... را با حرف اول کوچک  در فریم ها بنویسید. مثال
\providetranslation{Theorem}{قضیه}%{\large \bf قضیه}
\providetranslation{Definition}{تعریف}
\providetranslation{Example}{مثال}
\providetranslation{Lemma}{لم}
\providetranslation{Corollary}{نتیجه}
\providetranslation{Solution}{پاسخ}
\providetranslation{Problem}{مسئله}
\providetranslation{Fact}{حقیقت}
%
\newtheorem{proposition}[theorem]{گزاره}
%++++++++++++++++++++++++++++++ @Tex_Ahmadi
\raggedleft
%%%%%%%%
\newcommand*{\co}[1]{\nast\textcolor{blue}{#1}}
%%%%
\title{یک اسلاید نمونه}
\subtitle{زیرعنوان اسلاید}
\author[مجتبیٰ احمدی]{\co{مجتبیٰ احمدی}
\\[.2cm]
\textcolor{red}{{\xb استاد راهنما}}:
\\[.2cm]
\co{جناب آقای}
}
\institute{دانشگاه: پیام‌نو‌ر مشگین شهر}
\date{\today}
%\date{28 بهمن 1395}
%%%%%%%%%%
\logo{\includegraphics[scale=.02]{logo1.png}}
\newcommand{\nologo}{\setbeamertemplate{logo}{}}

\begin{document}

{\nologo
\begin{frame}[plain,noframenumbering]
%برای عدم نمایش نوار بالایی و پایین در فرم آپشن plain و برای اینکه جزء شماره اسلاید محسوب نشود آپشن فرم noframenumbering قرار داده شده است.
\centerline{\includegraphics[width=\paperwidth,height=\paperheight]{besm.jpg}}
\end{frame}}
%%%%%%%%%%%%%%
\begin{frame}
\begin{picture}(0,0)
\put (0,-20){\centerline{\includegraphics[width=1.3cm]{logo.jpg}}}
\end{picture}
\vskip 0.1 in
\maketitle
\end{frame}
%%%%%%%%%%%%%%
\begin{frame}{فهرست مطالب}
\tableofcontents
\end{frame}
%%%%%%%%%%%%%%%%%%
\section{قسمتی که در اینجاست}
\subsection{زیرقسمتی که در اینجاست}
\subsubsection{زیرزیرقسمتی که در اینجاست}

\begin{frame}
\frametitle{عنوان}
\framesubtitle{زیرعنوان}
\begin{definition}
این یک تعریف است 
\begin{align}
2x - 5y &=  8 \\ 
3x + 9y &=  -12
\end{align}
\begin{itemize}\raggedright
\item
 آیتم اوّل
\item
 آیتم دوّم
\end{itemize}
\ptext[1]
\begin{itemize}\itemindent=2em
\begin{LTRitems}
\item $A=\{1, 2, 3, \dots, 20\}$
\item $5\in A$
\end{LTRitems}
\end{itemize}
\end{definition}

\end{frame}
%%%%%%%%%%%%%%%%%%%
\begin{frame}
\ptext[1]
\begin{enumerate}\itemindent=2em
\item 
این یک متن است که در اینجا قرار می‌دهیم.
\end{enumerate}
\pause

\begin{itemize}\itemindent=2em
\item 
این یک متن است که در اینجا قرار می‌دهیم.
\end{itemize}

\begin{description}\itemindent=4em
\item[آزمایش]
این یک متن است که در اینجا قرار می‌دهیم.
\end{description}

\end{frame}
%%%%%%%%%%%%%%%%%%%%%%%%
\begin{frame}

\footnote{این یک زیرنویس پارسی است.}%

\LTRfootnote{This is a latin footnote.}%

\RTLfootnote{این هم زیرنویس پارسی دیگری است.}%

\footnote{این یک زیرنویس پارسی است.}

\end{frame}
%%%%%%%%%%%%%%%%%
\begin{frame}
\ptext[1]
\begin{definition}
این یک تعریف است
\end{definition}
\pause

\begin{theorem}
این یک قضیه است
\end{theorem}
\pause

\begin{proof}
این یک اثبات است.
\end{proof}
\pause

\begin{example}
این یک مثال است
\end{example}
\end{frame}
%++++++++
\begin{frame}{شمارش محیط‌ها}
\begin{definition}
این یک تعریف است
\begin{figure}
\centering
\includegraphics[scale=.02]{logo1.png} 
\caption{لوگو}
\end{figure}
\end{definition}
\pause

\begin{theorem}
این یک قضیه است
\end{theorem}
\pause

\begin{proof}
این یک اثبات است.
\end{proof}

\end{frame}
% 7
\begin{frame}
\begin{example}
این یک مثال است
\end{example}
\begin{corollary}
متن
\end{corollary}

\begin{proposition}
تست
\end{proposition}
\begin{lemma}
متن
\end{lemma}
\end{frame}
%++++++++
%%%%%%%%%%%%%%%%%%%%%%%
\section{بخش بعدی}
\subsection{زیربخش بعدی}
\subsubsection{زیرزیربخش بعدی}
\begin{frame}
\begin{columns} 
 \column{.5\textwidth}
ستون شماره ۱
\begin{latin}
This is a test as you can see
\end{latin}
 \column{.2\textwidth}
ستون شماره ۲
\column{0.3\textwidth}
ستون شماره ۳
\end{columns}

\begin{block}{فرمول ریاضی}
\begin{equation}
\sum_{i=1}^{n} i = \frac{n(n+1)}{2}
\end{equation}
\end{block}

\end{frame}
%%%%%%%%%%%%%%%%%%%%%%%%%%%
\begin{frame}{عنوان اسلاید این صفحه  که در اینجا قرار می‌گیرد}
\begin{example}
این یک مثال است.
\end{example}


\begin{definition}
این یک تعریف است.
\begin{itemize}\raggedright
\item 1
\item 2
\end{itemize}
\end{definition}

\begin{proof}
این یک اثبات است که در اینجا نوشته می‌شود و فقط مقدار بیشتر متن و متن که نوشته می‌شود تا بتوانیم به سطر بعدی برویم.
\end{proof}

\end{frame}
%%%%%%%%%%%%%%%%%%%%%%%%%
\begin{frame}[fragile]{کد متلب}
\textcolor{red}{توجه}:
برای نوشتن کد برنامه‌ای نیاز هست که آپشن فرم رو 
\lr{fragile}
قرار دهیم.
\begin{latin}
\begin{lstlisting}[frame=single,rulecolor=\color{magenta},numbers=left,numberstyle=\tiny]
%==================================================
% Author:M.Ahmadi                   date:1394.10.20
%                 http://matlabp.ir
%==================================================
% The numbers 1 to 9, to form a rhombus.
clc
n=9;
for i=1:n
  for k=n-1:-1:i
    fprintf(' ');
  end;
  for j=1:i
    fprintf('%d',j);
  end;
  for j=i-1:-1:1
    fprintf('%d',j);
  end;
  fprintf('\n');
end;
\end{lstlisting}
\end{latin}
\end{frame}
%%%%%%%%%
\setbeamertemplate{frametitle continuation}{\insertcontinuationcount}
%آپشن allowframebreaks برای شکستن مراجع قرار داده شده
\begin{frame}[allowframebreaks]{منابع}
\setbeamertemplate{bibliography item}{\insertbiblabel}
\begin{latin}
\begin{thebibliography}{9}
\bibitem{1}
Mojtaba Ahmadi.
 \newblock {The Math books and Examples of \LaTeX in Weblog}.
 \newblock \texttt{http://ahmadi1386.persianblog.ir/}, 1993/04/15.
\bibitem{2}
 Vafa Khalighi.
 \newblock {The XePersian Package, Persian for \LaTeX2e over \XeTeX}.
 \newblock \texttt{http://ctan.org/pkg/xepersian}, 2008.
 \bibitem{3}
 This is a test for break.  This is a test for break. 1
\bibitem{4}
 This is a test for break.  This is a test for break. 2
 \bibitem{5}
 This is a test for break.  This is a test for break. 3
 \bibitem{6}
 This is a test for break.  This is a test for break. 4
 \bibitem{7}
 This is a test for break.  This is a test for break. 5
\bibitem{8}
 This is a test for break.  This is a test for break. 6
 \bibitem{9}
 This is a test for break.  This is a test for break. 7
 \bibitem{10}
 This is a test for break.  This is a test for break. 8
 \bibitem{11}
 This is a test for break.  This is a test for break. 9
 \bibitem{12}
 This is a test for break.  This is a test for break. 10
 \bibitem{13}
 This is a test for break.  This is a test for break. 11
\end{thebibliography}
\end{latin}

\end{frame}
%
\usebackgroundtemplate{\includegraphics[width=\paperwidth,height=\paperheight]{20.jpg} }
\nologo
\begin{frame}
\begin{center}
\Huge {\nas
\textcolor{red}{با آرزوی موفقیت}}
\end{center}
\end{frame}

\end{document}
